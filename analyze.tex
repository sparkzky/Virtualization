% !TeX program = xelatex
\documentclass[analyze]{mancls}
\usepackage{fontspec}

\title{DragonOS 虚拟化子系统 v0.1.10}
\author{DragonOS 开发团队}
\date{2025年4月}

\begin{document}

\section{项目介绍}

\subsection{背景与目标}
随着云计算、边缘计算及容器技术的飞速发展,虚拟化技术已成为现代操作系统不可或缺的核心能力。DragonOS 作为一个面向云原生基础设施的自研操作系统,其虚拟化子系统旨在提供类 KVM 的高性能虚拟机支持,具备灵活性、可扩展性和安全性。核心目标如下:

\begin{itemize}
    \item 支持基于 Intel VT-x 和 EPT 技术的硬件辅助虚拟化,实现高性能虚拟机环境;
    \item 构建模块化虚拟化栈,便于未来移植至 RISC-V 等其他架构;
    \item 提供用户态 API,用于虚拟机创建、销毁、运行控制、VCPU 调度等;
    \item 提供软硬件结合的内存与中断虚拟化支持,降低 VMExit 开销;
    \item 实现最小可信计算基(TCB),降低虚拟化逻辑对内核其他子系统的依赖。
\end{itemize}

\subsection{总体架构}
DragonOS 的虚拟化子系统采用分层式架构,以清晰的职责划分提升可维护性与安全性。主要层次包括:

\begin{itemize}
    \item \textbf{虚拟化入口层(virt/vm)}:为用户空间提供统一的 ioctl 接口,封装 VM 与 VCPU 的创建与生命周期控制逻辑;
    \item \textbf{VMM 管理层(kvm\_host)}:管理每个虚拟机实例的运行状态,调度虚拟 CPU,模拟中断控制器(LAPIC)行为;
    \item \textbf{内存虚拟化层(mmu)}:封装对 EPT/NPT 的支持,映射 Guest 物理地址到 Host 地址空间;
    \item \textbf{硬件操作抽象层(vmx, cpuid, mtrr)}:封装 VMX 操作、控制寄存器管理、CPU 特征枚举等底层细节;
    \item \textbf{体系结构相关适配层(arch/x86\_64/vm)}:提供体系结构特定支持,包括 VMX 初始化、VMCS 管理、异常退出处理等。
\end{itemize}

\subsection{功能特点}
DragonOS 虚拟化子系统具备以下关键功能特性:

\begin{enumerate}
    \item \textbf{多核 VCPU 支持}:通过 VCPU 实例管理,支持每个虚拟机运行多个并行处理核心;
    \item \textbf{硬件辅助虚拟化}:基于 Intel VMX 指令集,使用 VMCS 结构管理虚拟机状态,降低指令模拟开销;
    \item \textbf{内存隔离与嵌套分页}:支持 EPT,实现 Guest 物理地址到 Host 地址的快速映射,提高内存访问效率;
    \item \textbf{中断与 APIC 模拟}:实现虚拟 LAPIC,支持 Guest 中断注入、定时器模拟等功能;
    \item \textbf{虚拟机运行时控制接口}:提供用于 VM 启动、停止、VCPU 运行、退出状态处理等的 API;
    \item \textbf{灵活的模块化设计}:各功能独立封装,便于后续引入 AMD SVM、SEV、IO-MMU 支持;
    \item \textbf{面向用户空间的交互接口}:暴露 ioctl 控制点,实现类 Linux KVM 的控制模型,兼容已有工具链。
\end{enumerate}

\subsection{当前实现状态}
当前版本(v0.1.10)已经实现了如下组件:

\begin{itemize}
    \item 基于 VMX 的 Guest 创建与退出处理流程(vmx/exit.rs, vmx/vmenter.S);
    \item EPT 建表逻辑和动态内存映射管理(vmx/ept/mod.rs);
    \item 支持 VMCS 初始化与寄存器字段加载/存储(vmx/vmcs);
    \item 虚拟机生命周期和多 VCPU 管理逻辑(virt/vm/kvm\_dev.rs, vcpu.rs);
    \item Guest CPUID 支持(cpuid.rs);
    \item 虚拟中断控制器 LAPIC 模拟(kvm\_host/lapic.rs);
    \item Guest 内存管理与映射维护(mmu/kvm\_mmu.rs, mmu\_internal.rs);
    \item 用户态控制路径(ioctl 分发与执行);
    \item 基于 MTRR 寄存器模拟处理部分物理内存类型标记(mtrr.rs)。
\end{itemize}

\subsection{未来规划与展望}
为满足高密度虚拟化部署与安全计算需求,DragonOS 虚拟化子系统计划未来扩展以下方向:

\begin{itemize}
    \item 支持 AMD-V 架构与 SVM 操作路径;
    \item 引入 GIC/APIC 可选模拟层,兼容 ARM 架构与异构平台;
    \item 加入用户态 I/O 虚拟化框架,实现 virtio 等设备支持;
    \item 集成 SEV(Secure Encrypted Virtualization)支持,提升 VM 的运行时加密保护能力;
    \item 实现中断重映射(Interrupt Remapping)与 IO-MMU 支持,增强设备直通能力;
    \item 与 DragonOS 容器运行时进行深度集成,实现轻量级虚拟容器混合运行;
    \item 提供 qemu-lite 或 rust-hypervisor 工具链支持,便于调试和测试。
\end{itemize}

\section{x86\_64 平台虚拟化模块分析}

\subsection{体系结构适配层(arch/x86\_64/vm)}
本模块承担 DragonOS 虚拟化子系统与具体硬件架构的粘合职责,主要包括对 VMX(Intel VT-x)的支持实现、特权寄存器操作封装、异常处理、中断向量配置等。核心组件包括:

\begin{itemize}
    \item \textbf{vmx.rs}:该模块定义 VMX 初始化与启用流程,包括检查 CPU 支持、设置控制寄存器(CR0、CR4)、加载 VMXON 区域与 VMCS 模板;
    \item \textbf{vmenter.S}:汇编实现虚拟机进入与退出的逻辑封装。通过 VMLAUNCH/VMPTRLD/VMRESUME 进入 Guest,上下文切换通过栈帧及 VMCS 管理;
    \item \textbf{exit.rs}:处理 VMExit 的原因分类与分发,例如 CPUID、HLT、IO、MSR 访问等;
    \item \textbf{vmcs.rs}:封装对 VMCS 的字段读写接口,包括 VM-entry/exit 控制、Guest/Host 状态字段管理;
    \item \textbf{vmx\_vmcs\_layout.rs}:定义 VMCS 结构中的各类字段编码,便于后续结构化访问。
\end{itemize}

此外,该目录还实现了对中断处理器 LAPIC 的虚拟化支持,捕捉 Guest 中断请求并注入,模拟本地中断计时器。

\subsection{VMM 管理模块(virt/vm)}
该模块为整个虚拟化子系统的控制中心,负责管理每一个 VM 实例的生命周期和调度其对应的 VCPU。其结构组织与 Linux KVM 相似:

\begin{itemize}
    \item \textbf{kvm\_dev.rs}:实现类 Unix 设备模型,封装 VCPU/VM 的创建、运行、配置接口。提供 ioctl 实现,供用户空间管理;
    \item \textbf{vcpu.rs}:定义 VCPU 的调度结构,负责执行虚拟 CPU 运行循环,包括 Guest 执行、VMExit 处理、上下文恢复等;
    \item \textbf{kvm\_run.rs}:封装 VMExit 信息与用户空间通信结构体,支持 CPUID 返回、异常传递、MMIO 模拟等。
\end{itemize}

每个虚拟机由一个 `KvmVm` 实例表示,其内部持有多个 `Vcpu` 实例。调度循环采用阻塞式运行,当 VMExit 发生时返回控制至用户空间。

\subsection{内存虚拟化模块(mmu)}
为实现高性能的虚拟内存访问,DragonOS 借助 Intel EPT 技术完成从 Guest 虚拟地址到 Host 物理地址的两级映射:

\begin{itemize}
    \item \textbf{kvm\_mmu.rs}:管理 Guest 物理页的映射关系,实现 EPT 页表的动态分配与更新;
    \item \textbf{mmu\_internal.rs}:提供页表构建的底层实现,如构造 4 级 EPT 页表项、处理页权限(RWX)、执行 TLB 刷新等;
    \item \textbf{mmu\_flush.rs}:定义用于 VM 内存失效操作(如 EPT INVL操作)以确保一致性;
    \item \textbf{mtrr.rs}:模拟 MTRR 寄存器,用于兼容 Guest 操作系统对不同内存区域类型的访问需求(UC/WC/WB)。
\end{itemize}

该模块支持动态映射、延迟建表、Write Protect 与执行保护功能,为 Guest 提供高效而安全的内存空间。

\subsection{中断与设备虚拟化支持}
中断处理为虚拟化系统的关键路径之一。DragonOS 中当前实现了本地 APIC 的仿真逻辑,并具备以下能力:

\begin{itemize}
    \item \textbf{lapic.rs}:模拟本地 APIC,支持中断计时器、本地中断注入、EOI 发送等;
    \item \textbf{inject.rs}:封装中断注入逻辑,VMExit 后根据优先级与 ISR 寄存器,决定是否注入外部或本地中断;
    \item \textbf{interrupt.rs}:定义 VCPU 级别的中断状态机,协调硬中断与软件异常的注入时机。
\end{itemize}

尽管当前尚未实现完整的 I/O APIC 或 MSI/MSI-X 模拟,但框架已为后续 PCI 设备直通、virtio 支持做好铺垫。

\subsection{用户空间接口与控制平面}
通过类 Unix 字符设备接口,DragonOS 提供一组 ioctl 操作支持用户空间创建、配置和运行虚拟机:

\begin{itemize}
    \item \textbf{open/create}:初始化 VM/VCPU 结构,分配所需资源;
    \item \textbf{KVM\_RUN}:启动 VCPU 执行循环,直到 VMExit 发生;
    \item \textbf{KVM\_SET\_CPUID, KVM\_SET\_REGS}:配置 VCPU 初始状态;
    \item \textbf{KVM\_SET\_USER\_MEM\_REGION}:注册 Guest 可访问内存区域;
    \item \textbf{KVM\_GET\_VCPU\_EVENT, KVM\_INJECT\_IRQ}:同步中断与异常。
\end{itemize}

该设计兼容通用工具链(如 QEMU),也方便未来 Rust-native 虚拟机管理工具的开发。

\subsection{模块间协作机制}
DragonOS 中各虚拟化模块之间通过结构化接口进行松耦合协作:

\begin{itemize}
    \item VCPU 每次运行(via KVM\_RUN)时,先由 VMM 调用 VMX 模块进行 Guest 切换;
    \item VMExit 发生后,交由 \texttt{exit.rs} 匹配 Exit 原因,并调用相应模块(CPUID、MMIO、LAPIC)处理;
    \item EPT 页表在 VCPU 创建时构建,支持在运行时动态更新;
    \item LAPIC 在 VM 初始化时创建,配合 Inject 模块在 VMExit 后响应中断注入;
    \item 用户态通过 ioctl 设置 VM 状态,底层通过 kvm\_host 模块转发至架构特定实现。
\end{itemize}

该分层模型极大增强了系统的模块化与可测试性,也便于后续扩展更多特性。


\section{通用虚拟化管理模块分析}

DragonOS 虚拟化子系统不仅依赖底层的硬件支持(如 VT-x 和 EPT),更需要在架构无关层面实现一套完整的虚拟化对象管理框架。该部分的核心职责包括虚拟机与虚拟 CPU 的生命周期控制、运行调度、资源管理与用户接口封装。其位置处于硬件抽象层与用户接口之间,起到“控制面”与“调度面”的桥梁作用。

\subsection{虚拟机结构管理(KvmVm)}
虚拟机(VM)是 DragonOS 虚拟化中的第一层抽象实体,其创建与销毁通过 `/dev/kvm` 对应的类 Unix 设备接口完成。在内部,VM 由 `KvmVm` 结构体表示,其主要职责包括:

\begin{itemize}
    \item 管理虚拟内存区域:包括注册、映射、解除用户内存区域;
    \item 管理 VCPU 集合:每个 VM 可包含多个 VCPU 实例,由内部分配唯一 VCPU ID;
    \item 保持 VM 配置状态:如 CPUID 模板、IRQ 路由表、中断控制策略等;
    \item 持有架构相关 VM 状态:如 VMCS、EPT 页表根指针等;
    \item 协调运行同步:使用内部锁或 RCU 实现多线程访问一致性。
\end{itemize}

`KvmVm` 是大多数 ioctl 操作的上下文容器,例如 `KVM\_SET\_USER\_MEMORY\_REGION` 会直接修改该结构的内存映射表。

\subsection{虚拟 CPU 控制器(Vcpu)}
每个 VCPU 代表 Guest OS 中的一个逻辑 CPU 核心,其核心结构为 `Vcpu`,负责:

\begin{itemize}
    \item 保存运行上下文:通用寄存器、控制寄存器、段寄存器等;
    \item 管理中断注入:通过 `InterruptController` 实现中断状态调度;
    \item 调用 VMM 接口运行:通过 `run()` 函数进入虚拟机执行循环;
    \item 处理 VMExit:根据退出原因调用调度分发器;
    \item 提供线程绑定能力:每个 VCPU 可绑定到固定 Host 线程执行。
\end{itemize}

VCPU 生命周期由 VM 管理器创建,并在 `KVM\_RUN` 命令触发后进入主循环,循环包含:
\begin{enumerate}
    \item 设置当前 VCPU 的上下文;
    \item 调用 arch 层的 VMLAUNCH / VMRESUME;
    \item 发生 VMExit 时处理并返回用户态。
\end{enumerate}

\subsection{运行调度与 VMExit 分发}
DragonOS 的虚拟化运行主循环(run-loop)采用阻塞式设计:每次运行由用户态触发,直到 VMExit 才返回。该循环主要在 `vcpu.rs` 中实现,运行机制如下:

\begin{itemize}
    \item `vcpu.run()` 调用 arch 层 `enter\_guest()` 封装;
    \item VMExit 原因通过 VMCS 返回,交由 `exit\_handler` 模块处理;
    \item 对于常见原因(CPUID、HLT、IO),使用 `match` 分发;
    \item 部分需要用户态处理的事件通过共享内存(`KvmRun`)写入;
    \item 响应后再次进入 Guest,直到下一次 Exit 或终止。
\end{itemize}

该机制支持精细化事件控制,并可扩展更多事件类型(如 I/O、Hypercall、PF 等)。

\subsection{KVM 设备接口实现(kvm\_dev.rs)}
为兼容标准管理工具(如 QEMU)和自定义控制程序,DragonOS 实现了类 Linux KVM 的设备接口逻辑,其主要实现位于 `kvm\_dev.rs`,提供如下 ioctl 入口:

\begin{itemize}
    \item \textbf{KVM\_CREATE\_VM}:创建并返回一个 VM 实例;
    \item \textbf{KVM\_CREATE\_VCPU}:为 VM 实例创建 VCPU;
    \item \textbf{KVM\_RUN}:触发 VCPU 的运行逻辑;
    \item \textbf{KVM\_SET/GET\_REGS}:读写 VCPU 的通用寄存器组;
    \item \textbf{KVM\_SET\_USER\_MEMORY\_REGION}:向 VM 注册 Guest 物理内存段;
    \item \textbf{KVM\_SET\_CPUID2}:配置 CPUID 响应模板;
\end{itemize}

用户态与内核态通过共享内存区结构 `KvmRun` 实现状态传输,该机制支持高性能、零拷贝的双向通信。

\subsection{中断控制与状态同步}
虚拟化环境下的中断管理由 `interrupt.rs` 和 `inject.rs` 模块协调完成,职责包括:

\begin{itemize}
    \item 管理中断标志(IF 位)、中断窗口;
    \item 实现中断注入机制(优先级、中断屏蔽);
    \item 与 LAPIC 交互决定注入窗口;
    \item VMExit 时判断是否需要立刻注入中断或等待下一次;
    \item 支持软件中断与外部硬中断统一调度。
\end{itemize}

未来将引入 I/O APIC 模拟、中断路由表、MSI 向量映射等特性。

\subsection{模块协作流程总览}
综上,DragonOS 的通用虚拟化管理模块通过以下协作流程完成 VM 的运行管理:

\begin{enumerate}
    \item 用户态通过 `/dev/kvm` 创建 VM 与 VCPU;
    \item 配置寄存器、内存映射和中断控制结构;
    \item 调用 `KVM\_RUN` 启动 VCPU 主循环;
    \item VCPU 运行时切换至 Guest 世界,触发 VMX 指令;
    \item VMExit 后交由 VCPU 执行分发与状态同步;
    \item 如果没有终止条件,继续 Resume Guest。
\end{enumerate}

模块间采用面向 trait 的接口调用,解耦良好,支持后续扩展如 VCPU 热插拔、NUMA 感知调度等。


\section{关键数据结构模块分析}

DragonOS 虚拟化子系统的实现核心在于若干关键结构体的组织和交互。它们定义了虚拟机、虚拟 CPU、内存映射和中断等对象的抽象模型,是实现 Guest 执行上下文与 Host 控制面的基础。

\subsection{Vm:虚拟机对象}
每个虚拟机对象包含以下字段:

\begin{itemize}
    \item \textbf{lock\_vm\_ref}:指向虚拟机的弱引用。
    \item \textbf{mm}:虚拟机的地址空间管理,表示内存管理(Arc 地址空间)。
    \item \textbf{max\_vcpus}:虚拟机的最大虚拟处理器数量。
    \item \textbf{created\_vcpus}:已创建的虚拟处理器数量。
    \item \textbf{online\_vcpus}:在线的虚拟处理器数量。
    \item \textbf{vcpus}:虚拟处理器集合,类型为 `HashMap<usize, Arc<LockedVirtCpu>>`。
    \item \textbf{memslots\_set}:内存槽集合,对应活动和非活动内存槽(类型为 `Vec<Vec<Arc<LockedVmMemSlotSet>>>`)。
    \item \textbf{memslots}:当前活动内存槽集合,类型为 `Vec<Arc<LockedVmMemSlotSet>>`。
    \item \textbf{nr\_memslot\_pages}:内存槽的总页数。
    \item \textbf{arch}:虚拟机的架构类型。
    \item \textbf{dirty\_ring\_size}:脏页环的大小。
    \item \textbf{nr\_memslots\_dirty\_logging}:脏内存槽的数量,用于脏页日志记录。
    \item \textbf{dirty\_ring\_with\_bitmap}:指示是否使用位图来表示脏页。
    \item \textbf{kvm\_vmx}:仅在 x86\_64 架构下存在,KVM 的 VMX 结构体(可选)。
    \item \textbf{mmu\_invalidate\_seq}:内存管理单元(MMU)无效化序列号,用于表示无效化操作的顺序。
\end{itemize}


该结构是多数 ioctl 操作的核心作用对象,构成虚拟化资源管理的顶层单位。

\subsection{Vcpu:虚拟处理器对象}
每个 VCPU 对应一个逻辑 CPU,其结构体如下:

\begin{itemize}
    \item \textbf{cpu}:处理器 ID。
    \item \textbf{kvm}:指向虚拟机的弱引用,表示与 KVM 的关联。
    \item \textbf{vcpu\_id}:VCPU 索引编号。
    \item \textbf{vcpu\_idx}:VCPU 索引,由 id 分配器获取。
    \item \textbf{pid}:关联的进程 ID(可选)。
    \item \textbf{preempted}:表示 VCPU 是否被抢占。
    \item \textbf{ready}:表示 VCPU 是否准备就绪。
    \item \textbf{last\_used\_slot}:最后使用的内存插槽(可选)。
    \item \textbf{stats\_id}:统计信息 ID。
    \item \textbf{pv\_time}:PV 时间映射缓存。
    \item \textbf{arch}:架构类型。
    \item \textbf{stat}:VCPU 的统计信息。
    \item \textbf{mode}:VCPU 模式。
    \item \textbf{guest\_debug}:Guest 调试信息。
    \item \textbf{private}:仅在 x86\_64 架构下存在,保存私有 VCPU 信息(可选)。
    \item \textbf{request}:记录 VCPU 请求。
    \item \textbf{run}:运行时上下文(可选)。
\end{itemize}


VCPU 是调度的最小单位,其生命周期由 KVM 管理器在用户态 ioctl 指令控制下构造、运行和销毁。

\subsection{UapiKvmRun:VCPU 运行上下文结构体}
`UapiKvmRun` 是一个与用户空间共享的结构体,用于描述一次 VCPU 执行的上下文,包括退出原因、标志位、寄存器状态等:

\begin{itemize}
    \item \textbf{request\_interrupt\_window}:请求中断窗口(即是否允许中断注入)。
    \item \textbf{immediate\_exit}:指示是否应立即退出 KVM。
    \item \textbf{padding1}:用于内存对齐的填充字节。
    \item \textbf{exit\_reason}:VCPU 退出的原因代码(如 IO、MMIO、HLT 等)。
    \item \textbf{ready\_for\_interrupt\_injection}:表示是否已准备好注入中断。
    \item \textbf{if\_flag}:中断标志(Interrupt Flag)。
    \item \textbf{flags}:其他扩展标志位。
    \item \textbf{cr8}:CR8 控制寄存器的当前值。
    \item \textbf{apic\_base}:本地 APIC 的基地址。
    \item \textbf{\_\_bindgen\_anon\_1}:联合体字段,包含多种退出原因的结构体,如 I/O、MMIO、故障等。
    \item \textbf{kvm\_valid\_regs}:表示哪些寄存器在用户空间是有效的(有效位掩码)。
    \item \textbf{kvm\_dirty\_regs}:表示哪些寄存器在用户空间已被修改(脏位掩码)。
    \item \textbf{s}:联合体字段 `uapi\_kvm\_run\_\_bindgen\_ty\_2`,用于 KVM 扩展结构体的附加信息。
\end{itemize}

\subsection{VMControlStructure:虚拟机控制结构体}
该结构体用于描述虚拟机控制块(VMCB、VMCS 等)或类似的控制数据区域,具有页对齐(4KB)并在结构布局上紧凑对齐。

\begin{itemize}
    \item \textbf{header}:控制结构头部标识,通常用于识别版本或结构类型。
    \item \textbf{abort}:异常或终止标志,用于指示控制结构是否被中断或出错。
    \item \textbf{data}:保留的数据区,大小为一页减去两个 \texttt{u32} 字段的大小。该区域用于存储控制数据或硬件需要的状态信息。
\end{itemize}

该结构体整体以 4096 字节(4KB)对齐,符合典型虚拟化硬件控制结构(如 VMCS/VMCB)的对齐与大小要求。


\subsection{小结}
以上结构体现 DragonOS 虚拟化子系统的核心抽象模型,它们构成了从用户空间请求发起、到 Guest 执行、再到状态返回的完整闭环。各结构之间通过引用、共享内存与 VMCS 映射完成高效联动,为上层虚拟机生命周期管理、性能优化和安全隔离提供坚实基础。

\section{亮点与创新点}

\begin{itemize}
  \item 模块化设计清晰:x86 平台相关部分与平台无关逻辑严格分离。
  \item VMX 支持全面:从 VMCS 构造到 VMExit 响应均具备可扩展能力。
  \item 多 VCPU 支持:结合 LAPIC 模拟具备 SMP 支持。
  \item 内存访问高性能路径优化:EPT 实现充分利用缓存结构。
  \item 构建良好的用户态接口:接口兼容 KVM,易于构建用户空间虚拟机管理工具。
\end{itemize}

\section{总结与展望}

DragonOS 虚拟化子系统已具备基本 VM 启动、运行、退出的完整流程,并封装了核心的地址转换、VMExit 管理与虚拟中断处理能力。未来可扩展方向包括:

\begin{itemize}
  \item 添加 AMD SVM 支持,实现平台通用性。
  \item 完善设备虚拟化支持(virtio)。
  \item 添加 Nested VM 支持,支持 L1/L2 嵌套虚拟化。
  \item 增强与 DragonOS Namespace/容器子系统的集成。
\end{itemize}

\end{document}

