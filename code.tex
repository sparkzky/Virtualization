% !TeX program = xelatex
\documentclass[code]{mancls}
\usepackage{fontspec}

\title{DragonOS虚拟化子系统 v0.1.10}

\begin{document}

\section{x86\_64架构下的虚拟化实现}

\subsection{mod.rs}
\lstinputlisting[style=codestyle]{DragonOS/kernel/src/arch/x86_64/vm/mod.rs}

\subsection{kvm\_host}

\subsubsection{\texttt{vcpu.rs}}
\lstinputlisting[firstline=53, lastline=152,style=codestyle]{DragonOS/kernel/src/arch/x86_64/vm/kvm_host/vcpu.rs}

\subsection{mmu}

\subsubsection{\texttt{kvm\_mmu.rs}}
\lstinputlisting[style=codestyle]{DragonOS/kernel/src/arch/x86_64/vm/mmu/kvm_mmu.rs}

\subsection{vmx}

\subsubsection{\texttt{exit.rs}}
\lstinputlisting[style=codestyle]{DragonOS/kernel/src/arch/x86_64/vm/vmx/exit.rs}

\subsubsection{\texttt{mod.rs}}
\lstinputlisting[firstline=3744, lastline=3759,style=codestyle]{DragonOS/kernel/src/arch/x86_64/vm/vmx/mod.rs}

\section{通用虚拟化实现}

\subsection{\texttt{kernel/src/virt/vm}}

\subsubsection{mod.rs}
\lstinputlisting[style=codestyle]{DragonOS/kernel/src/virt/vm/mod.rs}

\subsubsection{kvm\_dev.rs}
\lstinputlisting[style=codestyle]{DragonOS/kernel/src/virt/vm/kvm_dev.rs}

\subsubsection{user\_api.rs}
\lstinputlisting[style=codestyle]{DragonOS/kernel/src/virt/vm/user_api.rs}

\subsubsection{kvm\_host/mem.rs}
\lstinputlisting[style=codestyle]{DragonOS/kernel/src/virt/vm/kvm_host/mem.rs}

\subsubsection{kvm\_host/vcpu.rs}
\lstinputlisting[style=codestyle]{DragonOS/kernel/src/virt/vm/kvm_host/vcpu.rs}


\end{document}
