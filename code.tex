% !TeX program = xelatex
\documentclass[code]{mancls}
\usepackage{fontspec}

\title{DragonOS虚拟化子系统 v0.1.10}

\begin{document}

\section{x86\_64架构下的虚拟化实现}

\subsection{asm.rs}
\lstinputlisting[style=codestyle]{DragonOS/kernel/src/arch/x86_64/vm/asm.rs}

\subsection{cpuid.rs}
\lstinputlisting[style=codestyle]{DragonOS/kernel/src/arch/x86_64/vm/cpuid.rs}

\subsection{mem.rs}
\lstinputlisting[style=codestyle]{DragonOS/kernel/src/arch/x86_64/vm/mem.rs}

\subsection{mod.rs}
\lstinputlisting[style=codestyle]{DragonOS/kernel/src/arch/x86_64/vm/mod.rs}

\subsection{mtrr.rs}
\lstinputlisting[style=codestyle]{DragonOS/kernel/src/arch/x86_64/vm/mtrr.rs}

\subsection{uapi.rs}
\lstinputlisting[style=codestyle]{DragonOS/kernel/src/arch/x86_64/vm/uapi.rs}

\subsection{kvm\_host}

\subsubsection{\texttt{lapic.rs}}
\lstinputlisting[style=codestyle]{DragonOS/kernel/src/arch/x86_64/vm/kvm_host/lapic.rs}

\subsubsection{\texttt{mod.rs}}
\lstinputlisting[style=codestyle]{DragonOS/kernel/src/arch/x86_64/vm/kvm_host/mod.rs}

\subsubsection{\texttt{page.rs}}
\lstinputlisting[style=codestyle]{DragonOS/kernel/src/arch/x86_64/vm/kvm_host/page.rs}

\subsubsection{\texttt{vcpu.rs}}
\lstinputlisting[style=codestyle]{DragonOS/kernel/src/arch/x86_64/vm/kvm_host/vcpu.rs}

\subsection{mmu}

\subsubsection{\texttt{kvm\_mmu.rs}}
\lstinputlisting[style=codestyle]{DragonOS/kernel/src/arch/x86_64/vm/mmu/kvm_mmu.rs}


\subsubsection{\texttt{mmu\_internal.rs}}
\lstinputlisting[style=codestyle]{DragonOS/kernel/src/arch/x86_64/vm/mmu/mmu_internal.rs}


\subsubsection{\texttt{mmu\_mod.rs}}
\lstinputlisting[style=codestyle]{DragonOS/kernel/src/arch/x86_64/vm/mmu/mod.rs}

\subsection{vmx}

\subsubsection{\texttt{asm.rs}}
\lstinputlisting[style=codestyle]{DragonOS/kernel/src/arch/x86_64/vm/vmx/asm.rs}

\subsubsection{\texttt{capabilities.rs}}
\lstinputlisting[style=codestyle]{DragonOS/kernel/src/arch/x86_64/vm/vmx/capabilities.rs}

\subsubsection{\texttt{exit.rs}}
\lstinputlisting[style=codestyle]{DragonOS/kernel/src/arch/x86_64/vm/vmx/exit.rs}

\subsubsection{\texttt{mod.rs}}
\lstinputlisting[style=codestyle]{DragonOS/kernel/src/arch/x86_64/vm/vmx/mod.rs}

\subsubsection{\texttt{vmenter.S}}
\lstinputlisting[style=codestyle]{DragonOS/kernel/src/arch/x86_64/vm/vmx/vmenter.S}

\subsubsection{\texttt{vmcx/feat.rs}}
\lstinputlisting[style=codestyle]{DragonOS/kernel/src/arch/x86_64/vm/vmx/vmcs/feat.rs}

\subsubsection{\texttt{vmcx/mod.rs}}
\lstinputlisting[style=codestyle]{DragonOS/kernel/src/arch/x86_64/vm/vmx/vmcs/mod.rs}

\subsubsection{\texttt{vmx/ept/mod.rs}}
\lstinputlisting[style=codestyle]{DragonOS/kernel/src/arch/x86_64/vm/vmx/ept/mod.rs}


\section{通用虚拟化实现}

\subsection{\texttt{kernel/src/virt/vm}}

\subsubsection{mod.rs}
\lstinputlisting[style=codestyle]{DragonOS/kernel/src/virt/vm/mod.rs}

\subsubsection{kvm\_dev.rs}
\lstinputlisting[style=codestyle]{DragonOS/kernel/src/virt/vm/kvm_dev.rs}

\subsubsection{user\_api.rs}
\lstinputlisting[style=codestyle]{DragonOS/kernel/src/virt/vm/user_api.rs}

\subsubsection{kvm\_host/mem.rs}
\lstinputlisting[style=codestyle]{DragonOS/kernel/src/virt/vm/kvm_host/mem.rs}

\subsubsection{kvm\_host/mod.rs}
\lstinputlisting[style=codestyle]{DragonOS/kernel/src/virt/vm/kvm_host/mod.rs}

\subsubsection{kvm\_host/vcpu.rs}
\lstinputlisting[style=codestyle]{DragonOS/kernel/src/virt/vm/kvm_host/vcpu.rs}


\end{document}
